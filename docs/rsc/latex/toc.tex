\documentclass[12pt]{letter} 
\usepackage{geometry}
\geometry{top=1in, bottom=1in, left=.75in, right=.75in}
\usepackage{newtxtext,newtxmath} % Set Times New Roman font
\usepackage{microtype} % Improved kerning and spacing
\usepackage{longtable}  % Add this package for multi-page tables
\usepackage[utf8]{inputenc}
\usepackage[table,xcdraw]{xcolor}

\pagestyle{empty}

\begin{document}

\begin{letter}{}

\begin{center}
    \textbf{$cover.cover_department$} \\
    \textbf{$cover.cover_division$} \\
    \textbf{$cover.cover_name$} \\
    \textbf{$cover.cover_location$}
\end{center}

\renewcommand{\arraystretch}{1.2}  % Adjust line height (1.5 is just an example, can be tweaked)

\vspace{2em}
\makebox[0.5\textwidth]{\hrulefill}
\begin{flushleft}
    \begin{tabular}{ p{0.50\textwidth} | p{0.50\textwidth} }
      $if(document.multipleRespondents)$
        \textbf{In the Matter of:} & \textbf{\hspace{1em}File No.:} \\
      $else$
        \textbf{In the Matter of:} \\
      $endif$ 
      \vspace{.0em} & \\ 
      $for(respondents)$
        $if(document.multipleRespondents)$
          \textbf{$it.full_name$} & \textbf{\hspace{1em}$it.file_number$} \\
          $if(it.status)$ 
            \textit{\textbf{Respondents}} & \\
            \textbf{$it.status$} & \\
            $if(it.count)$ 
              & \\
            $endif$
          $endif$
        $else$
          \textbf{$it.full_name$} & \textbf{\hspace{1em}$it.file_number$} \\ \\
          $if(it.status)$ 
            \textit{\textbf{Respondent}} 
            \textbf{$it.status$}
            $if(it.count)$ 
              \vspace{.25em} & \\
            $endif$
          $endif$
        $endif$    
      $endfor$
    \end{tabular}
\end{flushleft}
\makebox[0.5\textwidth]{\hrulefill}
 
\vspace{1em}

\fontsize{20}{22}\selectfont % 20pt font size with 24pt line height

\begin{center}
    \underline{\textbf{TABLE OF CONTENTS}}
\end{center}

\vspace{1em}
 
\fontsize{14}{20}\selectfont % 20pt font size with 24pt line height

\begin{longtable}{|c|p{0.7\textwidth}|r|}
    \hline
    \textbf{TAB} & \textbf{DESCRIPTION OF EXHIBIT} & \textbf{PAGE(S)} \\
    \hline
\endfirsthead
    \hline
    \textbf{TAB} & \textbf{DESCRIPTION OF EXHIBIT} & \textbf{PAGE(S)} \\
    \hline
\endhead
    $for(contents)$
    \textbf{${it.letter}} & ${it.title} & ${it.pageRange} \\
    \hline
    $endfor$
\end{longtable}

\end{letter}

\end{document}
